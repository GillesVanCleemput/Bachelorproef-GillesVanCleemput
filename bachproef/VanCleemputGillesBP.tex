%===============================================================================
% LaTeX sjabloon voor de bachelorproef toegepaste informatica aan HOGENT
% Meer info op https://github.com/HoGentTIN/latex-hogent-report
%===============================================================================

\documentclass[dutch,dit,thesis]{hogentreport}

% TODO:
% - If necessary, replace the option `dit`' with your own department!
%   Valid entries are dbo, dbt, dgz, dit, dlo, dog, dsa, soa
% - If you write your thesis in English (remark: only possible after getting
%   explicit approval!), remove the option "dutch," or replace with "english".

\usepackage{lipsum} % For blind text, can be removed after adding actual content

%% Pictures to include in the text can be put in the graphics/ folder
\graphicspath{{graphics/}}

%% For source code highlighting, requires pygments to be installed
%% Compile with the -shell-escape flag!
\usepackage[section]{minted}
%% If you compile with the make_thesis.{bat,sh} script, use the following
%% import instead:
%% \usepackage[section,outputdir=../output]{minted}
\usemintedstyle{solarized-light}
\definecolor{bg}{RGB}{253,246,227} %% Set the background color of the codeframe

%% Change this line to edit the line numbering style:
\renewcommand{\theFancyVerbLine}{\ttfamily\scriptsize\arabic{FancyVerbLine}}

%% Macro definition to load external java source files with \javacode{filename}:
\newmintedfile[javacode]{java}{
    bgcolor=bg,
    fontfamily=tt,
    linenos=true,
    numberblanklines=true,
    numbersep=5pt,
    gobble=0,
    framesep=2mm,
    funcnamehighlighting=true,
    tabsize=4,
    obeytabs=false,
    breaklines=true,
    mathescape=false
    samepage=false,
    showspaces=false,
    showtabs =false,
    texcl=false,
}

% Other packages not already included can be imported here

%%---------- Document metadata -------------------------------------------------
% TODO: Replace this with your own information
\author{Ernst Aarden}
\supervisor{Dhr. F. Van Houte}
\cosupervisor{Mevr. S. Beeckman}
\title[Optionele ondertitel]%
    {Titel van de bachelorproef}
\academicyear{\advance\year by -1 \the\year--\advance\year by 1 \the\year}
\examperiod{1}
\degreesought{\IfLanguageName{dutch}{Professionele bachelor in de toegepaste informatica}{Bachelor of applied computer science}}
\partialthesis{false} %% To display 'in partial fulfilment'
%\institution{Internshipcompany BVBA.}

%% Add global exceptions to the hyphenation here
\hyphenation{back-slash}

%% The bibliography (style and settings are  found in hogentthesis.cls)
\addbibresource{bachproef.bib}            %% Bibliography file
\addbibresource{../voorstel/voorstel.bib} %% Bibliography research proposal
\defbibheading{bibempty}{}

%% Prevent empty pages for right-handed chapter starts in twoside mode
\renewcommand{\cleardoublepage}{\clearpage}

\renewcommand{\arraystretch}{1.2}

%% Content starts here.
\begin{document}

%---------- Front matter -------------------------------------------------------

\frontmatter

\hypersetup{pageanchor=false} %% Disable page numbering references
%% Render a Dutch outer title page if the main language is English
\IfLanguageName{english}{%
    %% If necessary, information can be changed here
    \degreesought{Professionele Bachelor toegepaste informatica}%
    \begin{otherlanguage}{dutch}%
       \maketitle%
    \end{otherlanguage}%
}{}

%% Generates title page content
\maketitle
\hypersetup{pageanchor=true}

%%=============================================================================
%% Voorwoord
%%=============================================================================

\chapter*{\IfLanguageName{dutch}{Woord vooraf}{Preface}}%
\label{ch:voorwoord}

%% TODO:
%% Het voorwoord is het enige deel van de bachelorproef waar je vanuit je
%% eigen standpunt (``ik-vorm'') mag schrijven. Je kan hier bv. motiveren
%% waarom jij het onderwerp wil bespreken.
%% Vergeet ook niet te bedanken wie je geholpen/gesteund/... heeft

\lipsum[1-2]
%%=============================================================================
%% Samenvatting
%%=============================================================================

% TODO: De "abstract" of samenvatting is een kernachtige (~ 1 blz. voor een
% thesis) synthese van het document.
%
% Een goede abstract biedt een kernachtig antwoord op volgende vragen:
%
% 1. Waarover gaat de bachelorproef?
% 2. Waarom heb je er over geschreven?
% 3. Hoe heb je het onderzoek uitgevoerd?
% 4. Wat waren de resultaten? Wat blijkt uit je onderzoek?
% 5. Wat betekenen je resultaten? Wat is de relevantie voor het werkveld?
%
% Daarom bestaat een abstract uit volgende componenten:
%
% - inleiding + kaderen thema
% - probleemstelling
% - (centrale) onderzoeksvraag
% - onderzoeksdoelstelling
% - methodologie
% - resultaten (beperk tot de belangrijkste, relevant voor de onderzoeksvraag)
% - conclusies, aanbevelingen, beperkingen
%
% LET OP! Een samenvatting is GEEN voorwoord!

%%---------- Nederlandse samenvatting -----------------------------------------
%
% TODO: Als je je bachelorproef in het Engels schrijft, moet je eerst een
% Nederlandse samenvatting invoegen. Haal daarvoor onderstaande code uit
% commentaar.
% Wie zijn bachelorproef in het Nederlands schrijft, kan dit negeren, de inhoud
% wordt niet in het document ingevoegd.

\IfLanguageName{english}{%
\selectlanguage{dutch}
\chapter*{Samenvatting}
\lipsum[1-4]
\selectlanguage{english}
}{}

%%---------- Samenvatting -----------------------------------------------------
% De samenvatting in de hoofdtaal van het document

\chapter*{\IfLanguageName{dutch}{Samenvatting}{Abstract}}

\lipsum[1-4]


%---------- Inhoud, lijst figuren, ... -----------------------------------------

\tableofcontents

% In a list of figures, the complete caption will be included. To prevent this,
% ALWAYS add a short description in the caption!
%
%  \caption[short description]{elaborate description}
%
% If you do, only the short description will be used in the list of figures

\listoffigures

% If you included tables and/or source code listings, uncomment the appropriate
% lines.
%\listoftables
%\listoflistings

% Als je een lijst van afkortingen of termen wil toevoegen, dan hoort die
% hier thuis. Gebruik bijvoorbeeld de ``glossaries'' package.
% https://www.overleaf.com/learn/latex/Glossaries

%---------- Kern ---------------------------------------------------------------

\mainmatter{}

% De eerste hoofdstukken van een bachelorproef zijn meestal een inleiding op
% het onderwerp, literatuurstudie en verantwoording methodologie.
% Aarzel niet om een meer beschrijvende titel aan deze hoofdstukken te geven of
% om bijvoorbeeld de inleiding en/of stand van zaken over meerdere hoofdstukken
% te verspreiden!

%%=============================================================================
%% Inleiding
%%=============================================================================

\chapter{\IfLanguageName{dutch}{Inleiding}{Introduction}}%
\label{ch:inleiding}

De inleiding moet de lezer net genoeg informatie verschaffen om het onderwerp te begrijpen en in te zien waarom de onderzoeksvraag de moeite waard is om te onderzoeken. In de inleiding ga je literatuurverwijzingen beperken, zodat de tekst vlot leesbaar blijft. Je kan de inleiding verder onderverdelen in secties als dit de tekst verduidelijkt. Zaken die aan bod kunnen komen in de inleiding~\autocite{Pollefliet2011}:

\begin{itemize}
  \item context, achtergrond
  \item afbakenen van het onderwerp
  \item verantwoording van het onderwerp, methodologie
  \item probleemstelling
  \item onderzoeksdoelstelling
  \item onderzoeksvraag
  \item \ldots
\end{itemize}

\section{\IfLanguageName{dutch}{Probleemstelling}{Problem Statement}}%
\label{sec:probleemstelling}

Uit je probleemstelling moet duidelijk zijn dat je onderzoek een meerwaarde heeft voor een concrete doelgroep. De doelgroep moet goed gedefinieerd en afgelijnd zijn. Doelgroepen als ``bedrijven,'' ``KMO's'', systeembeheerders, enz.~zijn nog te vaag. Als je een lijstje kan maken van de personen/organisaties die een meerwaarde zullen vinden in deze bachelorproef (dit is eigenlijk je steekproefkader), dan is dat een indicatie dat de doelgroep goed gedefinieerd is. Dit kan een enkel bedrijf zijn of zelfs één persoon (je co-promotor/opdrachtgever).

\section{\IfLanguageName{dutch}{Onderzoeksvraag}{Research question}}%
\label{sec:onderzoeksvraag}

Wees zo concreet mogelijk bij het formuleren van je onderzoeksvraag. Een onderzoeksvraag is trouwens iets waar nog niemand op dit moment een antwoord heeft (voor zover je kan nagaan). Het opzoeken van bestaande informatie (bv. ``welke tools bestaan er voor deze toepassing?'') is dus geen onderzoeksvraag. Je kan de onderzoeksvraag verder specifiëren in deelvragen. Bv.~als je onderzoek gaat over performantiemetingen, dan 

\section{\IfLanguageName{dutch}{Onderzoeksdoelstelling}{Research objective}}%
\label{sec:onderzoeksdoelstelling}

Wat is het beoogde resultaat van je bachelorproef? Wat zijn de criteria voor succes? Beschrijf die zo concreet mogelijk. Gaat het bv.\ om een proof-of-concept, een prototype, een verslag met aanbevelingen, een vergelijkende studie, enz.

\section{\IfLanguageName{dutch}{Opzet van deze bachelorproef}{Structure of this bachelor thesis}}%
\label{sec:opzet-bachelorproef}

% Het is gebruikelijk aan het einde van de inleiding een overzicht te
% geven van de opbouw van de rest van de tekst. Deze sectie bevat al een aanzet
% die je kan aanvullen/aanpassen in functie van je eigen tekst.

De rest van deze bachelorproef is als volgt opgebouwd:

In Hoofdstuk~\ref{ch:stand-van-zaken} wordt een overzicht gegeven van de stand van zaken binnen het onderzoeksdomein, op basis van een literatuurstudie.

In Hoofdstuk~\ref{ch:methodologie} wordt de methodologie toegelicht en worden de gebruikte onderzoekstechnieken besproken om een antwoord te kunnen formuleren op de onderzoeksvragen.

% TODO: Vul hier aan voor je eigen hoofstukken, één of twee zinnen per hoofdstuk

In Hoofdstuk~\ref{ch:conclusie}, tenslotte, wordt de conclusie gegeven en een antwoord geformuleerd op de onderzoeksvragen. Daarbij wordt ook een aanzet gegeven voor toekomstig onderzoek binnen dit domein.
\chapter{\IfLanguageName{dutch}{Stand van zaken}{State of the art}}%
\label{ch:stand-van-zaken}

% Tip: Begin elk hoofdstuk met een paragraaf inleiding die beschrijft hoe
% dit hoofdstuk past binnen het geheel van de bachelorproef. Geef in het
% bijzonder aan wat de link is met het vorige en volgende hoofdstuk.

% Pas na deze inleidende paragraaf komt de eerste sectiehoofding.

Dit hoofdstuk bevat je literatuurstudie. De inhoud gaat verder op de inleiding, maar zal het onderwerp van de bachelorproef *diepgaand* uitspitten. De bedoeling is dat de lezer na lezing van dit hoofdstuk helemaal op de hoogte is van de huidige stand van zaken (state-of-the-art) in het onderzoeksdomein. Iemand die niet vertrouwd is met het onderwerp, weet nu voldoende om de rest van het verhaal te kunnen volgen, zonder dat die er nog andere informatie moet over opzoeken \autocite{Pollefliet2011}.

Je verwijst bij elke bewering die je doet, vakterm die je introduceert, enz.\ naar je bronnen. In \LaTeX{} kan dat met het commando \texttt{$\backslash${textcite\{\}}} of \texttt{$\backslash${autocite\{\}}}. Als argument van het commando geef je de ``sleutel'' van een ``record'' in een bibliografische databank in het Bib\LaTeX{}-formaat (een tekstbestand). Als je expliciet naar de auteur verwijst in de zin (narratieve referentie), gebruik je \texttt{$\backslash${}textcite\{\}}. Soms is de auteursnaam niet expliciet een onderdeel van de zin, dan gebruik je \texttt{$\backslash${}autocite\{\}} (referentie tussen haakjes). Dit gebruik je bv.~bij een citaat, of om in het bijschrift van een overgenomen afbeelding, broncode, tabel, enz. te verwijzen naar de bron. In de volgende paragraaf een voorbeeld van elk.

\textcite{Knuth1998} schreef een van de standaardwerken over sorteer- en zoekalgoritmen. Experten zijn het erover eens dat cloud computing een interessante opportuniteit vormen, zowel voor gebruikers als voor dienstverleners op vlak van informatietechnologie~\autocite{Creeger2009}.

Let er ook op: het \texttt{cite}-commando voor de punt, dus binnen de zin. Je verwijst meteen naar een bron in de eerste zin die erop gebaseerd is, dus niet pas op het einde van een paragraaf.

\lipsum[7-20]

%%=============================================================================
%% Methodologie
%%=============================================================================

\chapter{\IfLanguageName{dutch}{Methodologie}{Methodology}}%
\label{ch:methodologie}

%% TODO: In dit hoofstuk geef je een korte toelichting over hoe je te werk bent
%% gegaan. Verdeel je onderzoek in grote fasen, en licht in elke fase toe wat
%% de doelstelling was, welke deliverables daar uit gekomen zijn, en welke
%% onderzoeksmethoden je daarbij toegepast hebt. Verantwoord waarom je
%% op deze manier te werk gegaan bent.
%% 
%% Voorbeelden van zulke fasen zijn: literatuurstudie, opstellen van een
%% requirements-analyse, opstellen long-list (bij vergelijkende studie),
%% selectie van geschikte tools (bij vergelijkende studie, "short-list"),
%% opzetten testopstelling/PoC, uitvoeren testen en verzamelen
%% van resultaten, analyse van resultaten, ...
%%
%% !!!!! LET OP !!!!!
%%
%% Het is uitdrukkelijk NIET de bedoeling dat je het grootste deel van de corpus
%% van je bachelorproef in dit hoofstuk verwerkt! Dit hoofdstuk is eerder een
%% kort overzicht van je plan van aanpak.
%%
%% Maak voor elke fase (behalve het literatuuronderzoek) een NIEUW HOOFDSTUK aan
%% en geef het een gepaste titel.

\lipsum[21-25]



% Voeg hier je eigen hoofdstukken toe die de ``corpus'' van je bachelorproef
% vormen. De structuur en titels hangen af van je eigen onderzoek. Je kan bv.
% elke fase in je onderzoek in een apart hoofdstuk bespreken.

%\input{...}
%\input{...}
%...

%%=============================================================================
%% Conclusie
%%=============================================================================

\chapter{Conclusie}%
\label{ch:conclusie}

% TODO: Trek een duidelijke conclusie, in de vorm van een antwoord op de
% onderzoeksvra(a)g(en). Wat was jouw bijdrage aan het onderzoeksdomein en
% hoe biedt dit meerwaarde aan het vakgebied/doelgroep? 
% Reflecteer kritisch over het resultaat. In Engelse teksten wordt deze sectie
% ``Discussion'' genoemd. Had je deze uitkomst verwacht? Zijn er zaken die nog
% niet duidelijk zijn?
% Heeft het onderzoek geleid tot nieuwe vragen die uitnodigen tot verder 
%onderzoek?

\lipsum[76-80]



%---------- Bijlagen -----------------------------------------------------------

\appendix

\chapter{Onderzoeksvoorstel}

Het onderwerp van deze bachelorproef is gebaseerd op een onderzoeksvoorstel dat vooraf werd beoordeeld door de promotor. Dat voorstel is opgenomen in deze bijlage.

%% TODO: 
%\section*{Samenvatting}

% Kopieer en plak hier de samenvatting (abstract) van je onderzoeksvoorstel.

% Verwijzing naar het bestand met de inhoud van het onderzoeksvoorstel
%---------- Inleiding ---------------------------------------------------------

\section{Introductie}%
\label{sec:introductie}

Dit bachelorproefschrift richt zich op de ontwikkeling van een proof of concept voor een systeem dat gepersonaliseerde video's genereert voor evenementbezoekers met behulp van gezichtsherkenningstechnologie. In tegenstelling tot conventionele methoden hoeven de bezoekers zelf geen foto's of video's te maken; de video wordt samengesteld op basis van de door de organisator vastgelegde beelden.

De onderzoeksgroep bestaat uit evenementenorganisatoren die streven naar het creëren van unieke herinneringen voor hun bezoekers. Het uiteindelijke resultaat van het onderzoek is een volledig uitgewerkte proof of concept, geïmplementeerd als een webapplicatie met integratie van een gezichtsherkenningstechnologie-API. Deze webapplicatie demonstreert effectief het vermogen om dynamisch gepersonaliseerde aftermovies te genereren op basis van evenementbeelden.

Dit voorstel beoogt niet alleen technologische innovatie, maar biedt ook een praktische oplossing voor evenementenorganisatoren die de deelnemerservaring willen verrijken. Het ultieme doel is het ontwikkelen van een gebruiksvriendelijke toepassing die zowel organisatoren als bezoekers ten goede komt.

%---------- Stand van zaken ---------------------------------------------------

\section{literatuurstudie}%
\label{sec:state-of-the-art}
In deze literatuurstudie onderzoeken we diverse papers die zich richten op het gebruik van artificiële intelligentie en gezichtsherkenning in videotoepassingen. Daarnaast gaan we dieper in op hoe nieuwe AI-tools kunnen worden ingezet voor geautomatiseerde videobewerking.

Als eerste bestuderen we het werk van de auteurs Nataliya Boyko, Oleg Basystiuk, en Nataliya Shakhovska \autocite{Boyko2018}. In hun studie vergelijken ze twee van de meest gebruikte en uitgebreide computer vision-algoritmes: OpenCV en Dlib. Deze vergelijking wordt uitgevoerd door middel van het ontwikkelen van een eenvoudige applicatie waarin beide algoritmes worden toegepast. De evaluatiecriteria omvatten de tijd die nodig is voor de uitvoering en het aantal iteraties dat vereist is om een bevredigend resultaat te bereiken.

De focus ligt op het meten van de prestaties van OpenCV en Dlib, waarbij de verwerkingstijd en de iteratieve benadering centrale aspecten zijn. Door de ontwikkeling van een praktische toepassing kunnen de auteurs de efficiëntie van beide algoritmes in een realistische context beoordelen. Deze evaluatie stelt hen in staat om inzicht te verschaffen in welk algoritme beter presteert op basis van de gegeven criteria. Deze bevindingen stellen ontwikkelaars in staat om weloverwogen beslissingen te nemen bij het selecteren van de meest geschikte computer vision-algoritmes voor hun toepassingen.

Bovendien hebben we een aanvullende paper onderzocht die nieuwe technieken en inzichten presenteert met betrekking tot het oplossen van het vraagstuk rond gezichtsherkenning in videomateriaal \autocite{Gorodnichy2020}. Het onderzoek legt eerst duidelijk het verschil vast tussen het herkennen van gezichten op foto's en in videomateriaal. De auteurs passen deze nieuwe methoden vervolgens toe op verschillende gebruiksscenario's, waaronder het analyseren van lage kwaliteit video's en live beeldfeeds. De uiteindelijke conclusie van de paper benadrukt het belangrijke inzicht dat als een persoon in staat is om iemand te herkennen, deze taak ook zeker haalbaar moet zijn voor computers.

In een andere studie werd gebruikgemaakt van gezichtsherkenningstechnologie in een laboratoriumomgeving om spontane aanmeldingen van individuen te registreren\autocite{Stallkamp2012}. De deelnemers waren niet op de hoogte van het gebruik van de technologie, wat resulteerde in natuurlijk gedrag zonder bewuste interactie met de camera. Dit leidde tot aanzienlijke variatie in facial appearance door verschillende lichtomstandigheden, houdingen en gelaatsuitdrukkingen. Het algoritme verwerkte deze gevarieerde dataset en leverde een zekerheidsscore op voor de identificatie van de personen.

De experimentele resultaten tonen aan dat het systeem een hoge mate van zekerheid bereikte en nauwkeurige identificaties maakte. Dit wijst op de effectiviteit van het gebruik van gezichtsherkenning in realistische, ongestuurde omgevingen waarbij mensen zich spontaan aanmelden. De bevindingen benadrukken de potentie van dergelijke technologieën in situaties waar deelnemers niet bewust rekening houden met het camerasysteem.


Daarnaast is er onderzoek verricht naar de toepassing van artificiële intelligentie voor het automatisch monteren van video's, met als doel evenementbezoekers een persoonlijke herinnering te bieden. In deze paper worden diverse AI-video-editing tools onderzocht en geëvalueerd\autocite{Soe2021}. Bovendien zijn de meningen van professionele videobewerkers verzameld om te beoordelen in hoeverre deze tools hen kunnen ondersteunen in hun werk.

Het onderzoek presenteert een overzicht van de nieuwste en meest geavanceerde AI-tools voor geautomatiseerde videobewerking, met inachtneming van de huidige technologische beperkingen. Hierdoor wordt niet alleen inzicht geboden in de mogelijkheden van AI in video-editing, maar ook in de mate waarin deze technologie kan bijdragen aan het vereenvoudigen en verbeteren van het werk van professionele videobewerkers.

Ten slotte werd een systematisch cartografisch onderzoek bekeken over het gebruik van artificiële intelligentie in de wereld van video-editing\autocite{IgorBieda}. Deze studie biedt een state-of-the-art overzicht van het onderwerp en identificeert gebieden waar verdere onderzoeksexploratie nodig is. Het vormt een uitgebreide samenvatting van de meest geavanceerde technologieën binnen AI voor videobewerking, waarbij de huidige stand van zaken wordt gepresenteerd en aangeeft waar de grenzen van onderzoek en ontwikkeling liggen.




%---------- Methodologie ------------------------------------------------------
\section{Methodologie}%
\label{sec:methodologie}

De methodologie van dit onderzoek omvat verschillende fasen, te beginnen met een grondige literatuurstudie. Deze studie heeft tot doel een diepgaand overzicht te verkrijgen van de huidige stand van zaken op het gebied van gezichtsherkenningstechnologie en het gebruik van AI-tools in videobewerking. Daarnaast zal een gedetailleerde vergelijking tussen diverse gezichtsherkenningstechnologieën worden uitgevoerd, waarbij de voor- en nadelen van elke technologie worden beschreven.

In deze fase worden de principes van gezichtsherkenningstechnologie uitgebreid uiteengezet, en wordt onderzocht hoe deze technologie effectief kan worden geïntegreerd in een webapplicatie. Dit omvat een diepgaande analyse van integratiemogelijkheden binnen een webomgeving. Na deze analyse zal specifieke aandacht worden besteed aan het onderzoeken van de toepasbaarheid van deze technologie voor het creëren van gepersonaliseerde aftermovies, met een verwachte tijdsduur van ongeveer 2 weken.

Na de literatuurstudie volgt de ontwikkeling van een proof of concept. Deze POC, een webapplicatie die gebruikmaakt van gezichtsherkenningstechnologie AI, zal gedurende ongeveer 8 weken worden gerealiseerd. De functionaliteiten van de POC omvatten het uploaden van een video van het evenement naar de webapplicatie. De gezichtsherkenningstechnologie identificeert vervolgens de gezichten van de bezoekers in de video en koppelt deze aan de bezoekers in de database, waardoor de POC gepersonaliseerde video's kan genereren.

Na de creatie van de POC volgt een uitgebreide testfase, waarin de functionaliteit van de POC wordt getest op echte evenementen. Tijdens deze fase wordt ook een evaluatie uitgevoerd van de code om mogelijke verbeteringen of aanpassingen te identificeren. De testfase heeft een geschatte duur van ongeveer 2 weken.

%---------- Verwachte resultaten ----------------------------------------------
\section{Verwacht resultaat, conclusie}%
\label{sec:verwachte_resultaten}
Aan het einde van dit onderzoek wordt een werkende proof of concept verwacht. Deze werkende proof of concept betreft een geïntegreerde webapplicatie met gezichtsherkenningstechnologie. Bezoekers van een evenement kunnen via deze applicatie een gepersonaliseerde video verkrijgen, waarbij ze bijvoorbeeld alleen beelden te zien krijgen van een specifiek gekozen persoon. Hierdoor kunnen de bezoekers op een meer persoonlijke manier herinneringen creëren aan het betreffende evenement. De beoogde resultaten omvatten een intuïtieve en toegankelijke gebruikersinterface, nauwkeurige gezichtsherkenning, en de mogelijkheid voor gebruikers om hun gewenste personalisaties eenvoudig te selecteren. De succesvolle implementatie van deze proof of concept zou niet alleen het functioneren van het systeem aantonen, maar ook zijn potentieel voor het verrijken van de evenementservaring voor individuele bezoekers demonstreren.



%%---------- Andere bijlagen --------------------------------------------------
% TODO: Voeg hier eventuele andere bijlagen toe. Bv. als je deze BP voor de
% tweede keer indient, een overzicht van de verbeteringen t.o.v. het origineel.
%\input{...}

%%---------- Backmatter, referentielijst ---------------------------------------

\backmatter{}

\setlength\bibitemsep{2pt} %% Add Some space between the bibliograpy entries
\printbibliography[heading=bibintoc]

\end{document}
