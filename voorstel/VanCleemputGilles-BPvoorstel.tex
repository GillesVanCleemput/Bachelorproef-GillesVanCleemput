%==============================================================================
% Sjabloon onderzoeksvoorstel bachproef
%==============================================================================
% Gebaseerd op document class `hogent-article'
% zie <https://github.com/HoGentTIN/latex-hogent-article>

% Voor een voorstel in het Engels: voeg de documentclass-optie [english] toe.
% Let op: kan enkel na toestemming van de bachelorproefcoördinator!
\documentclass{hogent-article}

% Invoegen bibliografiebestand
\addbibresource{voorstel.bib}

% Informatie over de opleiding, het vak en soort opdracht
\studyprogramme{Professionele bachelor toegepaste informatica}
\course{Bachelorproef}
\assignmenttype{Onderzoeksvoorstel}
% Voor een voorstel in het Engels, haal de volgende 3 regels uit commentaar
% \studyprogramme{Bachelor of applied information technology}
% \course{Bachelor thesis}
% \assignmenttype{Research proposal}

\academicyear{2023-2024} % TODO: pas het academiejaar aan

% TODO: Werktitel
\title{Hoe kan de integratie van gezichtsherkenningstechnologie bijdragen aan de ontwikkeling van gepersonaliseerde videos voor evenementbezoekers?}

% TODO: Studentnaam en emailadres invullen
\author{Gilles Van Cleemput}
\email{Gilles.vancleemput@student.hogent.be}

% TODO: Medestudent
% Gaat het om een bachelorproef in samenwerking met een student in een andere
% opleiding? Geef dan de naam en emailadres hier
% \author{Yasmine Alaoui (naam opleiding)}
% \email{yasmine.alaoui@student.hogent.be}

% TODO: Geef de co-promotor op
\supervisor[Co-promotor]{W. Himpe (Relivo, \href{mailto:willem@relivo.io}{willem@relivo.io})}

% Binnen welke specialisatierichting uit 3TI situeert dit onderzoek zich?
% Kies uit deze lijst:
%
% - Mobile \& Enterprise development
% - AI \& Data Engineering
% - Functional \& Business Analysis
% - System \& Network Administrator
% - Mainframe Expert
% - Als het onderzoek niet past binnen een van deze domeinen specifieer je deze
%   zelf
%
\specialisation{Mobile \& Enterprise development}
\keywords{Scheme, World Wide Web, $\lambda$-calculus}

\begin{document}

\begin{abstract}
  Dit onderzoeksvoorstel beoogt een proof of concept (POC) te ontwikkelen voor een systeem dat gepersonaliseerde aftermovies creëert via gezichtsherkenningstechnologie,
  specifiek gericht op evenementen, bruiloften en festivals. De POC, als webapplicatie met integratie van een gezichtsherkenningstechnologie,
  maakt het eenvoudig voor gebruikers om evenementenvideo's te uploaden. De technologie identificeert nauwkeurig gezichten van bezoekers en koppelt deze aan bijbehorende
  profielen in de database, waardoor op maat gemaakte aftermovies ontstaan.
  Het onderzoek richt zich op de doeltreffende integratie van geavanceerde technologie binnen de
  context van sociale evenementen en streeft naar een tastbaar resultaat in de vorm van een goed functionerende POC.
  De webapplicatie, met focus op eenvoud en gebruiksgemak, beoogt waarde toe te voegen voor zowel evenementorganisatoren als deelnemers.
  De verwachte uitkomst is een operationele POC die de potentie van het systeem aantoont en de weg vrijmaakt voor verdere ontwikkeling en implementatie.
\end{abstract}

\tableofcontents

% De hoofdtekst van het voorstel zit in een apart bestand, zodat het makkelijk
% kan opgenomen worden in de bijlagen van de bachelorproef zelf.
%---------- Inleiding ---------------------------------------------------------

\section{Introductie}%
\label{sec:introductie}

Dit bachelorproefschrift richt zich op de ontwikkeling van een proof of concept voor een systeem dat gepersonaliseerde video's genereert voor evenementbezoekers met behulp van gezichtsherkenningstechnologie. In tegenstelling tot conventionele methoden hoeven de bezoekers zelf geen foto's of video's te maken; de video wordt samengesteld op basis van de door de organisator vastgelegde beelden.

De onderzoeksgroep bestaat uit evenementenorganisatoren die streven naar het creëren van unieke herinneringen voor hun bezoekers. Het uiteindelijke resultaat van het onderzoek is een volledig uitgewerkte proof of concept, geïmplementeerd als een webapplicatie met integratie van een gezichtsherkenningstechnologie-API. Deze webapplicatie demonstreert effectief het vermogen om dynamisch gepersonaliseerde aftermovies te genereren op basis van evenementbeelden.

Dit voorstel beoogt niet alleen technologische innovatie, maar biedt ook een praktische oplossing voor evenementenorganisatoren die de deelnemerservaring willen verrijken. Het ultieme doel is het ontwikkelen van een gebruiksvriendelijke toepassing die zowel organisatoren als bezoekers ten goede komt.

%---------- Stand van zaken ---------------------------------------------------

\section{literatuurstudie}%
\label{sec:state-of-the-art}
In deze literatuurstudie onderzoeken we diverse papers die zich richten op het gebruik van artificiële intelligentie en gezichtsherkenning in videotoepassingen. Daarnaast gaan we dieper in op hoe nieuwe AI-tools kunnen worden ingezet voor geautomatiseerde videobewerking.

Als eerste bestuderen we het werk van de auteurs Nataliya Boyko, Oleg Basystiuk, en Nataliya Shakhovska \autocite{Boyko2018}. In hun studie vergelijken ze twee van de meest gebruikte en uitgebreide computer vision-algoritmes: OpenCV en Dlib. Deze vergelijking wordt uitgevoerd door middel van het ontwikkelen van een eenvoudige applicatie waarin beide algoritmes worden toegepast. De evaluatiecriteria omvatten de tijd die nodig is voor de uitvoering en het aantal iteraties dat vereist is om een bevredigend resultaat te bereiken.

De focus ligt op het meten van de prestaties van OpenCV en Dlib, waarbij de verwerkingstijd en de iteratieve benadering centrale aspecten zijn. Door de ontwikkeling van een praktische toepassing kunnen de auteurs de efficiëntie van beide algoritmes in een realistische context beoordelen. Deze evaluatie stelt hen in staat om inzicht te verschaffen in welk algoritme beter presteert op basis van de gegeven criteria. Deze bevindingen stellen ontwikkelaars in staat om weloverwogen beslissingen te nemen bij het selecteren van de meest geschikte computer vision-algoritmes voor hun toepassingen.

Bovendien hebben we een aanvullende paper onderzocht die nieuwe technieken en inzichten presenteert met betrekking tot het oplossen van het vraagstuk rond gezichtsherkenning in videomateriaal \autocite{Gorodnichy2020}. Het onderzoek legt eerst duidelijk het verschil vast tussen het herkennen van gezichten op foto's en in videomateriaal. De auteurs passen deze nieuwe methoden vervolgens toe op verschillende gebruiksscenario's, waaronder het analyseren van lage kwaliteit video's en live beeldfeeds. De uiteindelijke conclusie van de paper benadrukt het belangrijke inzicht dat als een persoon in staat is om iemand te herkennen, deze taak ook zeker haalbaar moet zijn voor computers.

In een andere studie werd gebruikgemaakt van gezichtsherkenningstechnologie in een laboratoriumomgeving om spontane aanmeldingen van individuen te registreren\autocite{Stallkamp2012}. De deelnemers waren niet op de hoogte van het gebruik van de technologie, wat resulteerde in natuurlijk gedrag zonder bewuste interactie met de camera. Dit leidde tot aanzienlijke variatie in facial appearance door verschillende lichtomstandigheden, houdingen en gelaatsuitdrukkingen. Het algoritme verwerkte deze gevarieerde dataset en leverde een zekerheidsscore op voor de identificatie van de personen.

De experimentele resultaten tonen aan dat het systeem een hoge mate van zekerheid bereikte en nauwkeurige identificaties maakte. Dit wijst op de effectiviteit van het gebruik van gezichtsherkenning in realistische, ongestuurde omgevingen waarbij mensen zich spontaan aanmelden. De bevindingen benadrukken de potentie van dergelijke technologieën in situaties waar deelnemers niet bewust rekening houden met het camerasysteem.


Daarnaast is er onderzoek verricht naar de toepassing van artificiële intelligentie voor het automatisch monteren van video's, met als doel evenementbezoekers een persoonlijke herinnering te bieden. In deze paper worden diverse AI-video-editing tools onderzocht en geëvalueerd\autocite{Soe2021}. Bovendien zijn de meningen van professionele videobewerkers verzameld om te beoordelen in hoeverre deze tools hen kunnen ondersteunen in hun werk.

Het onderzoek presenteert een overzicht van de nieuwste en meest geavanceerde AI-tools voor geautomatiseerde videobewerking, met inachtneming van de huidige technologische beperkingen. Hierdoor wordt niet alleen inzicht geboden in de mogelijkheden van AI in video-editing, maar ook in de mate waarin deze technologie kan bijdragen aan het vereenvoudigen en verbeteren van het werk van professionele videobewerkers.

Ten slotte werd een systematisch cartografisch onderzoek bekeken over het gebruik van artificiële intelligentie in de wereld van video-editing\autocite{IgorBieda}. Deze studie biedt een state-of-the-art overzicht van het onderwerp en identificeert gebieden waar verdere onderzoeksexploratie nodig is. Het vormt een uitgebreide samenvatting van de meest geavanceerde technologieën binnen AI voor videobewerking, waarbij de huidige stand van zaken wordt gepresenteerd en aangeeft waar de grenzen van onderzoek en ontwikkeling liggen.




%---------- Methodologie ------------------------------------------------------
\section{Methodologie}%
\label{sec:methodologie}

De methodologie van dit onderzoek omvat verschillende fasen, te beginnen met een grondige literatuurstudie. Deze studie heeft tot doel een diepgaand overzicht te verkrijgen van de huidige stand van zaken op het gebied van gezichtsherkenningstechnologie en het gebruik van AI-tools in videobewerking. Daarnaast zal een gedetailleerde vergelijking tussen diverse gezichtsherkenningstechnologieën worden uitgevoerd, waarbij de voor- en nadelen van elke technologie worden beschreven.

In deze fase worden de principes van gezichtsherkenningstechnologie uitgebreid uiteengezet, en wordt onderzocht hoe deze technologie effectief kan worden geïntegreerd in een webapplicatie. Dit omvat een diepgaande analyse van integratiemogelijkheden binnen een webomgeving. Na deze analyse zal specifieke aandacht worden besteed aan het onderzoeken van de toepasbaarheid van deze technologie voor het creëren van gepersonaliseerde aftermovies, met een verwachte tijdsduur van ongeveer 2 weken.

Na de literatuurstudie volgt de ontwikkeling van een proof of concept. Deze POC, een webapplicatie die gebruikmaakt van gezichtsherkenningstechnologie AI, zal gedurende ongeveer 8 weken worden gerealiseerd. De functionaliteiten van de POC omvatten het uploaden van een video van het evenement naar de webapplicatie. De gezichtsherkenningstechnologie identificeert vervolgens de gezichten van de bezoekers in de video en koppelt deze aan de bezoekers in de database, waardoor de POC gepersonaliseerde video's kan genereren.

Na de creatie van de POC volgt een uitgebreide testfase, waarin de functionaliteit van de POC wordt getest op echte evenementen. Tijdens deze fase wordt ook een evaluatie uitgevoerd van de code om mogelijke verbeteringen of aanpassingen te identificeren. De testfase heeft een geschatte duur van ongeveer 2 weken.

%---------- Verwachte resultaten ----------------------------------------------
\section{Verwacht resultaat, conclusie}%
\label{sec:verwachte_resultaten}
Aan het einde van dit onderzoek wordt een werkende proof of concept verwacht. Deze werkende proof of concept betreft een geïntegreerde webapplicatie met gezichtsherkenningstechnologie. Bezoekers van een evenement kunnen via deze applicatie een gepersonaliseerde video verkrijgen, waarbij ze bijvoorbeeld alleen beelden te zien krijgen van een specifiek gekozen persoon. Hierdoor kunnen de bezoekers op een meer persoonlijke manier herinneringen creëren aan het betreffende evenement. De beoogde resultaten omvatten een intuïtieve en toegankelijke gebruikersinterface, nauwkeurige gezichtsherkenning, en de mogelijkheid voor gebruikers om hun gewenste personalisaties eenvoudig te selecteren. De succesvolle implementatie van deze proof of concept zou niet alleen het functioneren van het systeem aantonen, maar ook zijn potentieel voor het verrijken van de evenementservaring voor individuele bezoekers demonstreren.



\printbibliography[heading=bibintoc]

\end{document}