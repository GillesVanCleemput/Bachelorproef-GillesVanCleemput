%---------- Inleiding ---------------------------------------------------------

\section{Introductie}%
\label{sec:introductie}

In dit bachelorproefs onderzoeksvoorstel wordt voorgesteld om een proof of concept te maken van een systeem dat gepersonaliseerde 
aftermovies maakt voor evenementbezoekers. De gepersonaliseerde aftermovies worden gemaakt door middel van gezichtsherkenningstechnologie.
De bezoekers hoeven zelf geen foto's of video's te maken; de aftermovie wordt gemaakt op basis van de beelden die de organisator van het evenement heeft gemaakt.

De doelgroep van het onderzoek zijn organisatoren van evenementen die een aftermovie willen maken voor hun bezoekers.
De aftermovie kan door de organisator gebruikt worden om de bezoekers een unieke herinnering aan het evenement te geven.

Het eindresultaat van het onderzoek is een volledig uitgewerkte proof of concept die de werking van het systeem kan demonstreren.
De proof of concept wordt gemaakt door middel van een webapplicatie. De webapplicatie maakt gebruik van een gezichtsherkenningstechnologie API.

%---------- Stand van zaken ---------------------------------------------------

\section{State-of-the-art}%
\label{sec:state-of-the-art}

Hier beschrijf je de \emph{state-of-the-art} rondom je gekozen onderzoeksdomein, d.w.z.\ een inleidende, doorlopende tekst over het onderzoeksdomein van je bachelorproef. Je steunt daarbij heel sterk op de professionele \emph{vakliteratuur}, en niet zozeer op populariserende teksten voor een breed publiek. Wat is de huidige stand van zaken in dit domein, en wat zijn nog eventuele open vragen (die misschien de aanleiding waren tot je onderzoeksvraag!)?

Je mag de titel van deze sectie ook aanpassen (literatuurstudie, stand van zaken, enz.). Zijn er al gelijkaardige onderzoeken gevoerd? Wat concluderen ze? Wat is het verschil met jouw onderzoek?

Verwijs bij elke introductie van een term of bewering over het domein naar de vakliteratuur, bijvoorbeeld~\autocite{Hykes2013}! Denk zeker goed na welke werken je refereert en waarom.

Draag zorg voor correcte literatuurverwijzingen! Een bronvermelding hoort thuis \emph{binnen} de zin waar je je op die bron baseert, dus niet er buiten! Maak meteen een verwijzing als je gebruik maakt van een bron. Doe dit dus \emph{niet} aan het einde van een lange paragraaf. Baseer nooit teveel aansluitende tekst op eenzelfde bron.

Als je informatie over bronnen verzamelt in JabRef, zorg er dan voor dat alle nodige info aanwezig is om de bron terug te vinden (zoals uitvoerig besproken in de lessen Research Methods).

% Voor literatuurverwijzingen zijn er twee belangrijke commando's:
% \autocite{KEY} => (Auteur, jaartal) Gebruik dit als de naam van de auteur
%   geen onderdeel is van de zin.
% \textcite{KEY} => Auteur (jaartal)  Gebruik dit als de auteursnaam wel een
%   functie heeft in de zin (bv. ``Uit onderzoek door Doll & Hill (1954) bleek
%   ...'')

Je mag deze sectie nog verder onderverdelen in subsecties als dit de structuur van de tekst kan verduidelijken.

%---------- Methodologie ------------------------------------------------------
\section{Methodologie}%
\label{sec:methodologie}

Het onderzoek zal zich in verschillende fasen voltrekken.
Als eerste zal er een diepgaande literatuurstudie worden uitgevoerd om de huidige stand van zaken op het gebied van gezichtsherkenningstechnologie in kaart te brengen.
Deze studie zal ook een vergelijking maken tussen verschillende gezichtsherkenningstechnologieën, waarbij de voor- en nadelen van elke technologie worden beschreven.
In deze fase zullen de principes van gezichtsherkenningstechnologie worden uiteengezet,
en er zal onderzocht worden hoe deze technologie geïntegreerd kan worden in een webapplicatie.
Na het onderzoeken van de integratiemogelijkheden binnen een webapplicatie,
zal er gekeken worden naar hoe deze technologie kan worden ingezet voor het maken van gepersonaliseerde aftermovies. Deze fase zal ongeveer 2 weken duren.

Na de literatuurstudie zal er een proof of concept (POC) worden ontwikkeld.
De POC zal een webapplicatie zijn die gebruikmaakt van gezichtsherkenningstechnologie AI.
De functionaliteiten van de POC omvatten het uploaden van een video van het evenement naar de webapplicatie,
waarbij de gezichtsherkenningstechnologie de gezichten van de bezoekers in de video herkent en koppelt aan de bezoekers in de database.
Hiermee kan de POC gepersonaliseerde aftermovies genereren. De ontwikkeling van de POC zal ongeveer 8 weken in beslag nemen.

Na de creatie van de POC zal een testfase worden uitgevoerd.
In deze fase zal de werking van de POC worden getest op echte evenementen,
en er zal ook een evaluatie plaatsvinden van de code om mogelijke verbeteringen of veranderingen te identificeren.
De testfase zal ongeveer 2 weken duren.

%---------- Verwachte resultaten ----------------------------------------------
\section{Verwacht resultaat, conclusie}%
\label{sec:verwachte_resultaten}
Aan het einde van dit bachelors onderzoek wordt een werkende proof of concept verwacht. Deze werkende proof of concept betreft een geïntegreerde webapplicatie met gezichtsherkenningstechnologie. Bezoekers van een evenement kunnen via deze applicatie een gepersonaliseerde video verkrijgen, waarbij ze bijvoorbeeld alleen beelden te zien krijgen van een specifiek gekozen persoon. Hierdoor kunnen de bezoekers op een meer persoonlijke manier herinneringen creëren aan het betreffende evenement. De beoogde resultaten omvatten een intuïtieve en toegankelijke gebruikersinterface, nauwkeurige gezichtsherkenning, en de mogelijkheid voor gebruikers om hun gewenste personalisaties eenvoudig te selecteren. De succesvolle implementatie van deze proof of concept zou niet alleen het functioneren van het systeem aantonen, maar ook zijn potentieel voor het verrijken van de evenementservaring voor individuele bezoekers demonstreren.

