%---------- Inleiding ---------------------------------------------------------

\section{Introductie}%
\label{sec:introductie}

In dit bachelorproefs onderzoeksvoorstel wordt voorgesteld om een proof of concept te maken van een systeem dat gepersonaliseerde 
aftermovies maakt voor evenementbezoekers. De gepersonaliseerde aftermovies worden gemaakt door middel van gezichtsherkenningstechnologie.
De bezoekers hoeven zelf geen foto's of video's te maken; de aftermovie wordt gemaakt op basis van de beelden die de organisator van het evenement heeft gemaakt.

De doelgroep van het onderzoek zijn organisatoren van evenementen die een aftermovie willen maken voor hun bezoekers.
De aftermovie kan door de organisator gebruikt worden om de bezoekers een unieke herinnering aan het evenement te geven.

Het eindresultaat van het onderzoek is een volledig uitgewerkte proof of concept die de werking van het systeem kan demonstreren.
De proof of concept wordt gemaakt door middel van een webapplicatie. De webapplicatie maakt gebruik van een gezichtsherkenningstechnologie API.

%---------- Stand van zaken ---------------------------------------------------

\section{literatuurstudie}%
\label{sec:state-of-the-art}
In deze literatuurstudie onderzoeken we diverse papers die zich richten op het gebruik van artificiële intelligentie en gezichtsherkenning in videotoepassingen. Daarnaast gaan we dieper in op hoe nieuwe AI-tools kunnen worden ingezet voor geautomatiseerde videobewerking.

Als eerste bestuderen we het werk van de auteurs Nataliya Boyko, Oleg Basystiuk, en Nataliya Shakhovska \autocite{Boyko2018}. In hun studie vergelijken ze twee van de meest gebruikte en uitgebreide computer vision-algoritmes: OpenCV en Dlib. Deze vergelijking wordt uitgevoerd door middel van het ontwikkelen van een eenvoudige applicatie waarin beide algoritmes worden toegepast. De evaluatiecriteria omvatten de tijd die nodig is voor de uitvoering en het aantal iteraties dat vereist is om een bevredigend resultaat te bereiken.

De focus ligt op het meten van de prestaties van OpenCV en Dlib, waarbij de verwerkingstijd en de iteratieve benadering centrale aspecten zijn. Door de ontwikkeling van een praktische toepassing kunnen de auteurs de efficiëntie van beide algoritmes in een realistische context beoordelen. Deze evaluatie stelt hen in staat om inzicht te verschaffen in welk algoritme beter presteert op basis van de gegeven criteria. Deze bevindingen stellen ontwikkelaars in staat om weloverwogen beslissingen te nemen bij het selecteren van de meest geschikte computer vision-algoritmes voor hun toepassingen.

Bovendien hebben we een aanvullende paper onderzocht die nieuwe technieken en inzichten presenteert met betrekking tot het oplossen van het vraagstuk rond gezichtsherkenning in videomateriaal \autocite{Gorodnichy2020}. Het onderzoek legt eerst duidelijk het verschil vast tussen het herkennen van gezichten op foto's en in videomateriaal. De auteurs passen deze nieuwe methoden vervolgens toe op verschillende gebruiksscenario's, waaronder het analyseren van lage kwaliteit video's en live beeldfeeds. De uiteindelijke conclusie van de paper benadrukt het belangrijke inzicht dat als een persoon in staat is om iemand te herkennen, deze taak ook zeker haalbaar moet zijn voor computers.

In een andere studie werd gebruikgemaakt van gezichtsherkenningstechnologie in een laboratoriumomgeving om spontane aanmeldingen van individuen te registreren\autocite{Stallkamp2012}. De deelnemers waren niet op de hoogte van het gebruik van de technologie, wat resulteerde in natuurlijk gedrag zonder bewuste interactie met de camera. Dit leidde tot aanzienlijke variatie in facial appearance door verschillende lichtomstandigheden, houdingen en gelaatsuitdrukkingen. Het algoritme verwerkte deze gevarieerde dataset en leverde een zekerheidsscore op voor de identificatie van de personen.

De experimentele resultaten tonen aan dat het systeem een hoge mate van zekerheid bereikte en nauwkeurige identificaties maakte. Dit wijst op de effectiviteit van het gebruik van gezichtsherkenning in realistische, ongestuurde omgevingen waarbij mensen zich spontaan aanmelden. De bevindingen benadrukken de potentie van dergelijke technologieën in situaties waar deelnemers niet bewust rekening houden met het camerasysteem.


Daarnaast is er onderzoek verricht naar de toepassing van artificiële intelligentie voor het automatisch monteren van video's, met als doel evenementbezoekers een persoonlijke herinnering te bieden. In deze paper worden diverse AI-video-editing tools onderzocht en geëvalueerd\autocite{Soe2021}. Bovendien zijn de meningen van professionele videobewerkers verzameld om te beoordelen in hoeverre deze tools hen kunnen ondersteunen in hun werk.

Het onderzoek presenteert een overzicht van de nieuwste en meest geavanceerde AI-tools voor geautomatiseerde videobewerking, met inachtneming van de huidige technologische beperkingen. Hierdoor wordt niet alleen inzicht geboden in de mogelijkheden van AI in video-editing, maar ook in de mate waarin deze technologie kan bijdragen aan het vereenvoudigen en verbeteren van het werk van professionele videobewerkers.

Ten slotte werd een systematisch cartografisch onderzoek bekeken over het gebruik van artificiële intelligentie in de wereld van video-editing\autocite{IgorBieda}. Deze studie biedt een state-of-the-art overzicht van het onderwerp en identificeert gebieden waar verdere onderzoeksexploratie nodig is. Het vormt een uitgebreide samenvatting van de meest geavanceerde technologieën binnen AI voor videobewerking, waarbij de huidige stand van zaken wordt gepresenteerd en aangeeft waar de grenzen van onderzoek en ontwikkeling liggen.




%---------- Methodologie ------------------------------------------------------
\section{Methodologie}%
\label{sec:methodologie}

Het onderzoek zal zich in verschillende fasen voltrekken.
Als eerste zal er een diepgaande literatuurstudie worden uitgevoerd om de huidige stand van zaken op het gebied van gezichtsherkenningstechnologie in kaart te brengen.
Deze studie zal ook een vergelijking maken tussen verschillende gezichtsherkenningstechnologieën, waarbij de voor- en nadelen van elke technologie worden beschreven.
In deze fase zullen de principes van gezichtsherkenningstechnologie worden uiteengezet,
en er zal onderzocht worden hoe deze technologie geïntegreerd kan worden in een webapplicatie.
Na het onderzoeken van de integratiemogelijkheden binnen een webapplicatie,
zal er gekeken worden naar hoe deze technologie kan worden ingezet voor het maken van gepersonaliseerde aftermovies. Deze fase zal ongeveer 2 weken duren.

Na de literatuurstudie zal er een proof of concept (POC) worden ontwikkeld.
De POC zal een webapplicatie zijn die gebruikmaakt van gezichtsherkenningstechnologie AI.
De functionaliteiten van de POC omvatten het uploaden van een video van het evenement naar de webapplicatie,
waarbij de gezichtsherkenningstechnologie de gezichten van de bezoekers in de video herkent en koppelt aan de bezoekers in de database.
Hiermee kan de POC gepersonaliseerde aftermovies genereren. De ontwikkeling van de POC zal ongeveer 8 weken in beslag nemen.

Na de creatie van de POC zal een testfase worden uitgevoerd.
In deze fase zal de werking van de POC worden getest op echte evenementen,
en er zal ook een evaluatie plaatsvinden van de code om mogelijke verbeteringen of veranderingen te identificeren.
De testfase zal ongeveer 2 weken duren.

%---------- Verwachte resultaten ----------------------------------------------
\section{Verwacht resultaat, conclusie}%
\label{sec:verwachte_resultaten}
Aan het einde van dit bachelors onderzoek wordt een werkende proof of concept verwacht. Deze werkende proof of concept betreft een geïntegreerde webapplicatie met gezichtsherkenningstechnologie. Bezoekers van een evenement kunnen via deze applicatie een gepersonaliseerde video verkrijgen, waarbij ze bijvoorbeeld alleen beelden te zien krijgen van een specifiek gekozen persoon. Hierdoor kunnen de bezoekers op een meer persoonlijke manier herinneringen creëren aan het betreffende evenement. De beoogde resultaten omvatten een intuïtieve en toegankelijke gebruikersinterface, nauwkeurige gezichtsherkenning, en de mogelijkheid voor gebruikers om hun gewenste personalisaties eenvoudig te selecteren. De succesvolle implementatie van deze proof of concept zou niet alleen het functioneren van het systeem aantonen, maar ook zijn potentieel voor het verrijken van de evenementservaring voor individuele bezoekers demonstreren.

